Para resolver este problema implementamos el algoritmo de Knuth-Morris-Pratt.

Since the two portions of the algorithm have, respectively, complexities of
$O(k)$ and $O(n)$, the complexity of the overall algorithm is $O(n+k)$.

(Hacer modelo en representacino del problema, luego algoritmo, luego analisis de
complejidad, luego relacionar las variables del analisis, que son del modelo,
con las del input, que son del problema, para obtener la complejidad en funcion
del input.)

\begin{verbatim}
The input consists of a number of test cases. Each test case is composed of
three lines, containing:
 * the length of the needle,
 * the needle itself,
 * the haystack.

The length of the needle is only limited by the memory available to
your program, so do not make any assumptions - instead, read the length
and allocate memory as needed. The haystack is not limited in size, which
implies that your program should not read the whole haystack at once. The
KMP algorithm is stream-based, i.e. it processes the haystack character by
character, so this is not a problem.


Create the failure function for target string. The failure function
is defined as follows, for i = 1 to m, where m is the length of
the target string: Let the target string be b_1...b_m. Then f(i) is
the length of the longest proper suffix of b_1...b_i which is a prefix
of the target string.

If you are trying to find the first occurence of a target string in
a stream of input characters and the i+1st character is wrong, you
need not start over looking for the beginning of the target from the
next received character on, because the last few characters received
might still be a valid head start on forming the target. The failure
function tells exactly how many characters head start you have.

In more colorful language, the idea behind this algorithm is: Don't
throw the baby out with the bathwater.

The code below was translated into C from psuedo code given on page
151 of Aho, Sethi, and Ullman. Unfortunately, the C-convention of
starting array indices from 0 makes the code harder to follow than it
should be. To make it easier, we never use f[0].


t = f[t];  /* This is the subtle part of the
algorithm. The target does not begin
t_0...t_t t_s. We could examine t_1...t_s,
t_2...t_s,... successively to see which (if any)
gives a valid prefix. But this is not
necessary. Say the longest valid prefix is
t_k...t_t t_s. Then t_k...t_t is also a valid
prefix. But f(t) gave the length of the
longest valid prefix of the form t_k...t_t.

Thus no k such that t_k..t_t is longer than
f(t) need be considered. Now iterate this
reasoning to conclude that only candidates
t_k...t_t of lengths f(t), f(f(t)),...
need be considered. */


Theoretical note: the assignment t = f[t] can be done at most m-1
times, where m is the length of the target string (although the
while loop can iterate more than once on given pass. For example,
if the target is abababb, then f(6)=4, f(f(6))=2, and f(f(f(6)))= 0.
The assignment is done 3 times when s = 6.) It is then easy to see that
the running time of the routine make_f is O(m).

Lemma. Let f be a non-negative integer valued function such that

f(1) = 0, and so that f is "continuous upwards", i.e., f(t+1) <= f(t) + 1.
Then

		__
  A = \ [ f(s) - f(s+1) ]+ <= (l-1),
		/__

  where a+ denotes the maximum of a and 0, and a- denotes the minimum
  of a and 0. Here and below, all sums run from 1 to l-1.

	Proof: Since a = (a+) + (a-) for any number a, we have
				__
	-f(l) = f(1) - f(l) = \ [ f(s) - f(s+1) ] =
				/__
		__
     A + \ [ f(s) - f(s+1) ]- >= A - (l-1).
		/__

	Using f(l) >= 0 and rearranging, we obtain the desired inequality.

								QED.

Consider now a given iteration of the 'for' loop with a given value
of s. Let n be the number of times the assignment t = f(t) is done.
			    (n) (n) (n)
Note that f(s+1) is either f (s) + 1, or 0 = f(s), where f denotes the
n-1 fold iterate of f. Since f(t) < t (only proper suffixes are
considered,) we have

	 (n)
	f (s) <= f(s) - n + 1.

Thus, in either case, f(s) - f(s+1) >= n. Now apply the lemma to obtain
the desired bound on the sum of n.

\end{verbatim}
