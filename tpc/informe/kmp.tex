Un string $s$ ($|s|=n$) ocurre con desplazamiento $p$ en un texto $t$ ($|t|=m$)
si los $n$ caracteres que siguen en $t$ desde la posición $p$ son iguales a los
de $s$. El problema consiste en encontrar todas los desplazamientos en que
ocurre un string $s$ en un texto $t$ (con $m$ en principio de tamaño ilimitado).

Lo resolvemos aplicando el algoritmo de Knuth-Morris-Pratt, que resuelve
precisamente este problema, por lo que su sola implementación redunda en
soluciones válidas, y no se hace necesario preprocesar los datos de entrada o de
salida para adaptarlos a los del problema en cuestión.

\subsection*{Detalles de implementación}

El enunciado sugiere buscar el string a medida que se lee el texto. Para
facilitar la primera implementación leimos tanto $s$ como $t$ completamente
antes de ejecutar KMP (mediante dos {\tt scanf}). Guardando $t$ en un arreglo de
1MB el algoritmo fue aceptado en SPOJ.

Mejoramos luego la implementación leyendo $t$ de a 256 bytes. La solución
también fue aceptada por SPOJ, el uso de memoria fue menor y el tiempo de
cómputo fue casi igual.

\subsection*{Análisis de complejidad}

La tabla de prefijos se genera en tiempo lineal sobre el tamaño del string a
buscar, $O(n)$; mientras que la búsqueda toma tiempo lineal sobre el tamaño
del texto, $O(m)$. La complejidad del algoritmo resulta entonces de $O(n+m)$.
