Un string $s$ ocurre con desplazamiento $p$ en un texto $t$ si los caracteres
de $s$ ocurren en $t$ desde $t[p]$. El problema consiste en encontrar todas
los desplazamientos en que ocurre un string $s$ en un texto $t$ ($|s|=n,
|t|=m$, $m$ en principio de tamaño ilimitado).

Aplicando el algoritmo de Knuth-Morris-Pratt, que resuelve precisamente
este problema, por lo que su implementación computa soluciones válidas,
y no se hace necesario preprocesar los datos de entrada o de salida para
adaptarlos a los del problema.

\subsection*{Detalles de implementación}

El enunciado sugiere buscar $s$ a medida que se lee $t$. Para facilitar la
primera implementación leimos completamente $s$ y $t$ antes de ejecutar KMP
(con dos {\tt scanf}). Guardando $t$ en un arreglo de 1MB el algoritmo fue
aceptado en SPOJ.

Mejoramos luego la implementación leyendo $t$ de a 256 bytes, de modo que
el tamaño de $t$ no sea una limitante de nuestro algoritmo. La solución
también fue aceptada por SPOJ, el uso de memoria resultó ser menor y el
tiempo de cómputo casi no varió.

\subsection*{Análisis de complejidad}

El algoritmo se compone de dos partes. Primero calcula la tabla de prefijos $s$,
lo que toma tiempo lineal sobre el tamaño de la entrada ($O(m)$). Luego hace la
búsqueda propiamente dicha, y toma tiempo lineal sobre el tamaño del texto ($O(m)$).

La complejidad del algoritmo resulta por tanto de $O(n+m)$.
