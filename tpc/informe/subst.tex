El problema consiste en determinar la cantidad de substrings distintos dada
una cadena de caracteres. El archivo de input del test contiene un conjunto de
cadenas y hay que calcular los substrings distintos para cada una de ellas.

Sabemos que si tenemos una cadena $s$, podemos obtener un substring $s_{ij}$
con $1 \le i \le j \le |s|$ conociendo el sufijo $s_i$ y tomando su prefijo
$s_{i_j}$. Además, resulta fácil ver que si tenemos dos sufijos $s_{k1}$ y 
$s_{k2}$ con $1 \le k1 \le k2 \le |s|$, si la primer letra del primer sufijo
es distinta a la primer letra del segundo sufijo, entonces los prefijos de esos
sufijos serán todos diferentes entre sí. En cambio si existe un prefijo en común
$p$ de longitud $1 \le |p| \le k1 \le k2$ los sufijos tendrán $|p|$ prefijos
en común, o sea que ambos presentan $|p|$ substrings en común. Por lo tanto, la
cantidad de substrings distintos que ambos sufijos generan será $|s_{k1}| + |s_{k1}| - |p|$.
Esta es la idea básica del algoritmo que resuelve el problema.

Conocemos una estructura que fue dada en clase y que ayuda mucho a resolver este
problema: el suffix array. Calculamos el suffix array de la cadena y el LCP array.
El objetivo principal es utilizar el LCP array para calcular la cantidad de substrings
que comparten dos sufijos.

Utilizamos una versión sencilla del suffix array que está basada en la
implementación presentada en el paper de Kasai. La idea del algoritmo es que
puede ordenar todos los sufijos comparando sólo 2 caracteres por iteración. En
cada iteración aprovecha la información obtenida en las iteraciones anteriores,
de forma que en cada paso conoce el doble de la información que en el anterior,
por lo que realiza un ordenamiento en $log(n)$ iteraciones.

Para calcular el LCP utilizamos el algoritmo de Sadakane que calcula el LCP
correspondiente al suffix array en tiempo lineal. El concepto principal del
algoritmo se basa en aprovechar que si el sufijo $s_{k1}$ y el $s_{k2}$ tienen en
común un prefijo $p$, los sufijos $s_{k1+1}$ y $s_{k2+1}$ tienen en común un prefijo
$p'$ de longitud $|p'| \ge |p|-1$.

Finalmente, para devolver la cantidad de substrings distintos calculamos la cantidad
total de substrings de la cadena que es $\displaystyle\frac{|s|*(|s|+1)}{2}$ y le restamos la
sumatoria del arreglo de LCP, que se corresponde con la cantidad total de substrings
repetidos.

\subsection*{Detalles de implementación}
%TODO: COMPLETAR CON LAS ESTRUCTURAS USADAS EN LA IMPLEMENTACION

\subsection*{Análisis de complejidad}

Ambos algoritmos los vimos en clase. El orden para calcular el suffix array es
$O( n*log^2(n) )$ y $O(n)$ para el LCP. La recorrida a través del LCP array tiene un
orden de $O(n)$. Por lo tanto el orden total del algoritmo es $O( n*log^2(n) )$.
