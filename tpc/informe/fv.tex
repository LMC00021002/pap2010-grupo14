Dado un arreglo $A$ de números no decrecientes, y un conjunto de pares $(i,j)$
(con $|A|=n$, $1 \le i \le j \le n$), el problema consiste en encontrar la
cantidad de veces que aparece el valor más frecuente en $A$ entre los índices de
cada consulta.

Procesamos los datos de entrada del problema de modo de poder obtener la
respuesta ejecutando {\sl Range Minimum Query}. RMQ indica el valor máximo o
mínimo entre dos índices de un arreglo, pero necesitamos conocer la frecuencia
con que aparece el más frecuente en ese rango. Para encontrarla, generamos un
arreglo tal que buscar el máximo entre los $i$ y $j$ de cada consulta mediante
RMQ sea la respuesta al problema.

[Explicamos el preprocesamiento aquí o en "Detalles"?]

\subsection*{Detalles de implementación}

Definimos las siguientes estructuras:

\begin{itemize}
  \item[\tt enteros] tiene largo $k$, siendo $k$ el número de elementos
  distintos en $A$, y en cada posición $i$ se almacena la cantidad de apariciones
  del iésimo elemento (distinto) de $A$.

  \item[\tt rangos] es un vector que contiene pares cuyos elementos denotan
  los índices de los subarreglos maximales con elementos iguales en $A$.

  \item[\tt mapIndices] es un diccionario sobre {\tt map}, que, dada una
  posición de $A$, indica la posición correspondiente en {\tt enteros}.
  {\tt mapIndices[$j$]}$=k \Leftrightarrow A[j]$ contiene al k-ésimo elemento
  distinto.
\end{itemize}

[FIXME: Explicar aquí que computamos la tabla RMQ (preprocesado: sparse table)
con "enteros".]

Dada una consulta $(i,j)$, computamos a qué pertenece cae cada índice (vector
{\tt rangos}). Si un mismo rango contiene a ambos, entonces el rango ``encierra''
a un conjunto de elementos iguales, por lo que el valor más frecuente es ese y
su cantidad de apariciones es $j-i$. Si $i$ y $j$ pertenencen a rangos distintos
(los llamamos $S$ y $T$), debemos encontrar la frecuencia del valor más
frecuente en $S,T$, y en los rangos $R_i$ que queden contenidos completamente
entre $S$ y $T$. Dicha frecuencia se computa en $O(1)$ en $S$ y $T$ [explicar
aritmética], y se la obtiene en los intervalos $R_i$ ejecutando RMQ. La
respuesta al problema es el máximo entre estos tres valores [demostrar mejor].

[Otros comentarios: Vector rangos es para un elemento en "enteros", de donde
hasta donde estaba en el original (i y j). Quedan k+2 rangos, con k los rangos
en el medio de S y T.]

\subsection*{Análisis de complejidad}

[Completar.]
