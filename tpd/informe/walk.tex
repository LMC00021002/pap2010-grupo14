\subsection*{Problema y modelo}

El problema consiste en encontrar el camino para que Alice visite caminando
a Bob minimizando la cantidad de subidas y bajadas que debe atravesar. La
descripción del terreno consiste en un mapa de alturas, compuesto por
polígonos que no se intersecan entre sí ni consigo mismos. Cada polígono
informa la altura del terreno sobre su perímetro.

La descripción que nos dan del terreno es parcial, pues no sabemos qué
alturas hay en el terreno en medio de cualquier par de curvas de altura
distintas. Por ello no podemos saber exactamente cuánto escala o desciende
Alice en su camino, pero podemos calcular el mínimo suponiendo un escenario
óptimo.

Presentamos un modelo simple en el que representamos los polígonos como
un conjunto de segmentos en los ejes $X\ Y$. Cada segmento tiene un valor de
altura dado por el enunciado. Alice y Bob se encuentran en las coordenadas
dadas por el enunciado (Alice en $(0,0)$, y Bob en $(100.000,0)$).

\subsection*{Solución}

Nuestra solución se basa en las siguientes ideas:

\begin{itemize}
\item Dado que no existe intersección entre polígonos, para cualquier par de
polígonos ocurre uno contiene al otro, o que son completamente disjuntos.

\item Distinguimos los polígonos según si Alice tiene que cruzarlos
obligatoriamente, o si puede evitarlos. De esta forma, ella cruza solamente
aquellos que necesariamente tiene que cruzar para ver a Bob. Un polígono es
evitable si existe algún camino entre las dos personas que no lo cruza.

\item Dado que no se puede saber cuáles son las alturas del terreno que
se ubica entre dos curvas de altura de distintos polígonos, la altura del
perímetro de cada polígono es independiente de la de los demás.
\end{itemize}

Cada vez que Alice cruza un polígono, la altura que debe escalar/bajar aumenta,
lo que significa que estos valores dependen de la cantidad y el orden de los
polígonos que cruza Alice. Logramos una cantidad mínima de polígonos cruzados si
respetamos el segundo principio.

Clasificar todos los polígonos según:

\begin{itemize}
\item Contiene solamente a Alice: para llegar a Bob, Alice va a tener que
cruzarlo, por lo que estos polígonos no son evitables.

\item Contiene solamente a Bob: Alice está fuera de este polígono, y necesita
cruzarlo, pues sino no podrá cumplir su objetivo. Estos polígonos no son
evitables.

\item No contiene a Alice ni a Bob: este polígono es evitable, es decir,
Alice no necesita cruzarlo para llegar a Bob, pues suponiendo que ella lo cruza
debe entrar y luego salir; llegaría a algún lugar al que le era posible llegar 
rodeando el polígono en cuestión sin necesidad de cruzarlo.

\item Contiene a ambos: Alice se puede mantener dentro de este polígono
y no necesita cruzarlo, de modo que también es evitable. En caso de que
Alice salga de este polígono, luego para llegar a Bob deberá cruzarlo. Si
consideramos el estado incial de Alice (justo antes de cruzar este polígono)
y su estado final (justo después de cruzarlo nuevamente) podemos decir que
siempre existe un camino entre estos estados inicial y final que no lo cruza
nunca ni atraviesa otros polígonos.
\end{itemize}

La solución al problema presentada ignora completamente los polígonos
evitables, dado que la altura de un polígono es independiente de la de los
demás. Consideremos entonces los polígonos no evitables.

Notemos que se puede definir un orden total entre los polígonos que contienen
solamente a Alice, y otro para los que contienen solamente a Bob. Consideramos
ordenado a una secuencia de polígonos tal que el primero es contenido por
todos los siguientes, y el último contiene a todos los anteriores. Esto
nos permite afirmar que existe una única solución para cada escenario del
problema, pues hay una sola forma de cruzar los polígonos que contienen a
Alice (de ``adentro'' hacia ``afuera''), y una sola forma de cruzar los que
contienen a Bob (de ``afuera'' hacia ``adentro''). Esta idea justifica la
solución presentada.

Dado un escenario, para cada polígono verificamos si contiene a Alice y
no a Bob o si contiene a Bob y no a Alice (ver sección ``Determinación
de pertenencia de un punto a un polígono''). En estos casos traducimos
el polígono a un par $(x_1, x_2, h)$ tal que $x_1$ y $x_2$ son dos puntos
distintos de intersección con el eje X, $x_1$ está a la izquierda del punto
que verificamos, $x_2$ a la derecha, y $h$ es la altura de las curvas de altura
del polígono. Podemos abstraernos de esta forma gracias a que los polígonos
están contenidos o totalmente disjuntos, de modo que para definir un orden
total entre los que están contenidos es suficiente con conocer cualquier par de
puntos de sus perímetros. En particular, nosotros tomamos la intersección
con el eje $X$ para describir los polígonos dado que debemos usar estos
puntos previamente para verificar la pertenencia de Alice (o de Bob).

Separamos en dos secuencias los pares que representan los polígonos que
cumplen que contienen a Alice y no a Bob, o que contienen a Bob y no a Alice. A
continuación ordenamos estas secuencias por contención de intervalos
(según la idea explicada en el párrafo anterior).

Luego recorremos de principio a fin la secuencia de pares que representan los
polígonos que contienen solamente a Alice, sumando en acumuladores de altura
escalada y altura bajada según corresponda. Lo recorremos en este orden
porque, para que Alice pueda salir de todos estos polígonos, primero tiene que
cruzar el más contenido y por último el que contiene a todos. Procedemos
análogamente para la secuencia de pares que representan los polígonos que
contienen sólo a Bob, pero recorriéndola de fin a principio. Lo recorremos
en este orden porque para que Alice pueda llegar a Bob debe cruzar primero el
polígono que contiene a todos, y por último el que es contenido por todos
(que es donde está Bob).

Finalmente devolvemos como resultado los acumuladores en el orden pedido
por el enunciado: primero la cantidad escalada, y luego la cantidad bajada.


\subsection*{Detalles de Implementación}

\subsection*{Análisis de complejidad}
