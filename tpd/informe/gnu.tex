\subsection*{Problema y solución}

El objetivo de este problema es simular reemplazos de caracteres en un
string, y calcular la cantidad de apariciones de un caracter determinado
en el string transformado. Los datos de entrada son las reglas $Q_i$ a aplicar,
un string $s$ que define la frecuencia inicial de cada caracter, el número de
veces $n$ a aplicar las reglas y el caracter $c$ del que se desea conocer su
frecuencia.

Por obvias razones de eficiencia, no ejecutamos los reemplazos sobre el
string dado, sino que los simulamos guardando en una matriz la frecuencia
de cada caracter. Esta matriz contiene inicialmente con las frecuencias
que aporta cada caracter según las reglas, y tiene la propiedad de que,
al multiplicarla por sí misma mantiene las frecuencias de haber aplicado
una vez las reglas dadas sobre el string.

Así, para el caso de ejemplo $A \rightarrow BAcX$, y dado el string de inicio
$ABCcXA$, ejemplos de matriz serán:

$$M =
\begin{array}{l}
\begin{matrix}
	\ \ A & B & C & X & c
\end{matrix} \\
\left(
	\begin{array}{ccccc}
	1 & 1 & 0 & 1 & 1 \\
	0 & 1 & 0 & 0 & 0 \\
	0 & 0 & 1 & 0 & 0 \\
	0 & 0 & 0 & 1 & 0 \\
	0 & 0 & 0 & 0 & 1 \\
	\end{array}
\right)
\hspace{1cm}
M^2 = \left(
\begin{array}{ccccc}
1 & 2 & 0 & 2 & 2 \\
0 & 1 & 0 & 0 & 0 \\
0 & 0 & 1 & 0 & 0 \\
0 & 0 & 0 & 1 & 0 \\
0 & 0 & 0 & 0 & 1 \\
\end{array}
\right)
\hspace{1cm}
M^3 = \left(
\begin{array}{ccccc}
1 & 3 & 0 & 3 & 3 \\
0 & 1 & 0 & 0 & 0 \\
0 & 0 & 1 & 0 & 0 \\
0 & 0 & 0 & 1 & 0 \\
0 & 0 & 0 & 0 & 1 \\
\end{array}
\right)
\end{array}
$$

Así, cada valor $M_{ij}^k$ representa la cantidad de caracteres $j$ que incorpora
el caracter $i$ en la aplicación de reglas número $k$. La frecuencia
de un caracter $c$ luego de $n$ aplicaciones resulta ser $\sum_i M_{ic}^n$.


\subsection*{Detalles de Implementación}

Dado que los valores se encuentran entre los números 33 y 126 de la tabla ASCII,
armamos matrices de dimensión $94\times 94$, que representan a todos los valores
de entrada posibles (al direccionar caracteres en la matriz restamos un offset
de 33 a su valor ASCII).


\subsection*{Análisis de complejidad}

Para calcular $M^e$ utilizamos el algoritmo de exponenciación binaria, basado en
que $x^e = (x^\frac e 2)^2$ (si $e$ es par), disminuyendo la cantidad de
multiplicaciones en forma logarítmica respecto del valor del exponente. Así, la
multiplicación de matrices (algoritmo de orden $N^3$, siendo $N$
la cantidad de columnas) se ejecuta $O(\log e)$ veces.

Rellenar la matriz con las $N$ {\sl queries} toma $O(N^3)$, y calcular la suma
final de frecuencias toma $O(N)$, por lo que la complejidad total resulta de
$O(N^3\log e + N)$.
