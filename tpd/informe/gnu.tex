\subsection*{Problema y solución}

El objetivo de este problema es calcular la cantidad de apariciones de una
letra en un string, que se obtiene luego de aplicar reglas de reemplazo.

Representamos las reglas como una matriz cuadrada, donde el valor $i,j$
representa la cantidad de caracteres $j$ que incorpora una aparición del
caracter $i$. Aplicar una vez las reglas equivale a multiplicar dicha matriz
consigo misma.


\subsection*{Detalles de Implementación}

\subsection*{Análisis de complejidad}

Exponentiation by Repeated Squaring runs in time proportional to the number of
bits in the binary representation of the exponent (length in bits is
approximately log of the value of the exponent). The naive implementation runs
much slower, in time proportional to the value of the exponent.
