\documentclass[12pt,spanish]{article}
\usepackage[spanish,activeacute]{babel}
\renewcommand\shorthandsspanish{}
\usepackage[utf8]{inputenc}
\usepackage[T1]{fontenc}
\usepackage{fancyhdr,a4wide,enumerate}
\usepackage{lastpage}
\usepackage{caratula}
\usepackage{amsthm,amsmath,pxfonts,amssymb}
\usepackage[ruled]{algorithm2e}
\usepackage[protrusion=true,expansion=true]{microtype}

\setlength{\headheight}{16pt}
\pagestyle{fancy}
\lhead{Trabajo Práctico C}
\rhead{Eugenio Costa, Emiliano González, Lisandro Sebrié}
\cfoot{\thepage\ de \pageref{LastPage}}
\renewcommand{\headrulewidth}{0.4pt}
\renewcommand{\footrulewidth}{0.4pt}

\theoremstyle{plain}
\newtheorem*{theorem}{Teorema}
\newtheorem*{lemma}{Lema}
\newtheorem*{proposition}{Proposición}
\newtheorem*{corollary}{Corolario}

\theoremstyle{definition}
\newtheorem*{definition}{Definición}

%\lstset{
%language=C++,
%tabsize=4,
%basicstyle=\footnotesize,
%breaklines=true,
%breakatwhitespace=true,
%}

\dontprintsemicolon
\linesnumberedhidden
\nocaptionofalgo
\SetKwIF{If}{ElseIf}{Else}{si}{luego}{sino si}{sino}{fin si}
\SetKw{Return}{devolver}
\SetKwFor{For}{para}{hacer}{fin para}
\SetKwFor{ForEach}{para cada}{hacer}{fin para}
\SetKwFor{ForAll}{repetir}{veces}{fin repetir}
\SetKwFor{While}{mientras}{hacer}{fin mientras}
\SetKwFor{Rep}{repetir}{veces}{fin repetir}
\SetKwRepeat{Repeat}{repetir}{hasta que}
\SetKw{KwTo}{hasta}
\SetKwInOut{Input}{entrada}
\SetKwInOut{Output}{salida}

\DeclareMathOperator{\set}{set}
\DeclareMathOperator{\sgn}{sgn}
\newcommand{\nat}{\mathbb{N}}
\newcommand{\integ}{\mathbb{Z}}

\begin{document}

\materia{Problemas, Algoritmos y Programación}
\titulo{Trabajo Práctico C}
\grupo{Grupo 14}
\integrante{Eugenio Costa}{839/06}{tutecosta@gmail.com}
\integrante{Emiliano González}{426/06}{xjesse\_jamesx@hotmail.com}
\integrante{Lisandro Sebrié}{358/03}{lsebrie@hotmail.com}

\maketitle

\newpage
\section*{32 -- ``A needle in the haystack''}
Un string $s$ ($|s|=n$) ocurre con desplazamiento $p$ en un texto $t$ ($|t|=m$)
si los $n$ caracteres que siguen en $t$ desde la posición $p$ son iguales a los
de $s$. El problema consiste en encontrar todas los desplazamientos en que
ocurre un string $s$ en un texto $t$ (con $m$ en principio de tamaño ilimitado).

Lo resolvemos aplicando el algoritmo de Knuth-Morris-Pratt, que resuelve
precisamente este problema, por lo que su sola implementación redunda en
soluciones válidas, y no se hace necesario preprocesar los datos de entrada o de
salida para adaptarlos a los del problema en cuestión.

\subsection*{Detalles de implementación}

El enunciado sugiere buscar el string a medida que se lee el texto. Para
facilitar la primera implementación leimos tanto $s$ como $t$ completamente
antes de ejecutar KMP (mediante dos {\tt scanf}). Guardando $t$ en un arreglo de
1MB el algoritmo fue aceptado en SPOJ.

Mejoramos luego la implementación leyendo $t$ de a 256 bytes. La solución
también fue aceptada por SPOJ, el uso de memoria fue menor y el tiempo de
cómputo fue casi igual.

\subsection*{Análisis de complejidad}

El algoritmo se compone de dos partes. Primero calcula la tabla de prefijos del
string a buscar, lo que toma tiempo lineal sobre el tamaño de la entrada
($O(m)$). Luego hace la búsqueda propiamente dicha, y toma tiempo lineal sobre
el tamaño del texto ($O(m)$).

La complejidad del algoritmo resulta por tanto de $O(n+m)$.


\newpage
\section*{11235 -- ``Frequent values''}
\subsection*{Solución}

Dada una secuencia $A$ de números no decrecientes, y un conjunto de pares $(i,j)$
(con $|A|=n$, $1 \le i \le j \le n$), el problema consiste en encontrar la
cantidad de veces que aparece el valor más frecuente en $A$ entre el intervalo
$(i,j)$ de índices de cada consulta.

Para solucionar el problema decidimos usar el algoritmo {\sl Range Minimum consulta}
(RMQ) visto en clase, pero una aplicación directa de dicho algoritmo para este problema
no es posible, pues RMQ encuentra un mínimo de una secuencia de números en un rango
acotado dado, y nosotros necesitamos un máximo, pero además este máximo no es
un entero de la secuencia, sino la cantidad de ocurrencias de un número.
Como vemos, aplicar RMQ al input no resolvería el problema.

Adaptar el {\sl Range Minimum consulta} a un {\sl Range Maximum consulta} es simple
y es un detalle de implementación que será explicado luego. Por ahora asumamos
que podemos calcular el {\sl Range Maximum consulta}.

Para adaptar la secuencia que nos provee el input la recorremos de forma
lineal compactando los enteros que son iguales y consecutivos por un entero
estrictamente mayor a $0$, que representa la cantidad de repeticiones de ese
entero en la secuencia original (sabemos que todos los enteros repetidos
aparecen de forma consecutiva porque la secuencia viene ordenada).

Notar que para la secuencia $s_t$ que resulta de transformar la secuencia $s$ vale
que $1 \le |s_t| \le |s|$. Por lo tanto, los índices del intervalo $(i,j)$
de la consulta del input pueden ser de valor mayor a $|s_t|$, por lo que necesitamos
una forma de mapear estos índices por índices de nuestra secuencia. Para esto
definimos $mapIndice$, que dado un índice de $A$, indica la posición correspondiente
en $s_t$ (llamémosla $k$), donde $s_{t_k}$ es la cantidad de veces que ocurre el
elemento $s_i$ en la secuencia $s$.

Dada esta transformación, es tentador utilizar RMQ directamente sobre $s_t$, pero
esto no resuelve el problema, pues una consulta podría ``partir'' intervalos.
Por ejemplo, para el input del enunciado
($s = [-1, -1, 1, 1, 1, 1, 3, 10, 10, 10]$, $s_t = [2, 4, 1, 3]$) y la consulta
$q=(2,3)$, $mapIndice(2) = 1$ y $mapIndice(3) = 2$, por lo tanto $RMQ_{s_t}(1,2)
=4$. Pero este resultado es incorrecto, pues debíamos devolver $1$, teniendo en
cuenta que no se seleccionan los intervalos completos.

Para solucionar esto, definimos la función auxiliar $mapRangos(i) = <k_1,k_2>$,
donde $1 \le k_1 \le k_2 \le L_s$ y $k_1$ es el índice de la primera aparición
del elemento $s_i$ en $s$, y $k_2$ es el índice de la última aparición de $s_i$ en $s$.

Ahora tenemos todas las estructuras necesarias para calcular la solución. Dado
un $s_t$ transformado de $s$ y una consulta $q = (i, j)$, tenemos una
secuencia de rangos $<k_{1_1}, k_{1_2}>, <k_{2_1}, k_{2_2}>, \ldots, <k_{m_1}, k_{m_2}>$
siendo $m = |s_t|$, y sabemos que $mapRangos(i) = <k_{p_1}, k_{p_2}>$ con
$p = mapIndice(i)$ y $mapRangos(j) = <k_{r_1}, k_{r_2}>$ con $r = mapIndice(j)$, $p \le r$.
La función que calcula el resultado será:

\[ FV(s, i, j) = \left\{ \begin{array}{ll}
                 j - i + 1 & \mbox{si $p = r$};\\
                 max\{ k_{p_2} - i + 1, j - k_{r_1} + 1, RMQ_{s_t}(p_2 + 1, r_1 - 1) \} & \mbox{si $p < r$}.\end{array} \right. \]
     
Intuitivamente esta función corre RMQ para todos los intervalos completos
que hay entre $i$ y $j$, y computa en forma particular el máximo de los
intervalos que quedan ``cortados'' por los índices, devolviendo como solución al
problema el máximo de los tres valores obtenidos.


\subsection*{Detalles de implementación}

Definimos para preprocesar y transformar los datos de la entrada las siguientes estructuras:

\begin{itemize}
  \item[\tt enteros] tiene largo $k$, siendo $k$ el número de elementos
  distintos en $s$. En cada posición $i$ se almacena la cantidad de apariciones
  del iésimo elemento (distinto) de $s$. Es el $s_t$ del modelo.

  \item[\tt rangos] es un vector que contiene pares cuyos elementos denotan
  los índices de los subarreglos maximales de elementos iguales en $s$. Cumple
  la función de mapRangos del modelo.

  \item[\tt mapIndices] es un vector que, dada una posición de $s$, indica la posición
  correspondiente en {\tt enteros}. Esto es: {\tt mapIndices[$j$]}$=i \Leftrightarrow s[j]$
  contiene al iésimo elemento distinto. Cumple la función de mapIndice del modelo.

  \item[\tt tabla] es la tabla RMQ (implementada sobre vectores de vectores).
  La obtenemos a partir del arreglo {\tt enteros} mediante el algoritmo {\sl
  Sparse Table}. Este algoritmo realiza un preprocesamiento en una matriz
  $M[0,n-1][0,\log n]$, siendo $M[i][j]$ el valor máximo en el subarreglo
  que empieza en $i$ de largo $2^j$ y $n$ el del enunciado. Para obtener el $RMQ_s(i, j)$ a partir
  de esta tabla, siendo $k = [log_2(j - i + 1)]$ se hace:
  \[ RMQ_s(i, j) = \left\{ \begin{array}{ll}
                 M[i][k] & \mbox{si $M[i][k] > M[j-2^k+1][k]$}\\
                 M[j-2^k+1][k] & \mbox{en otro caso}\end{array} \right. \]

  Para mayor información, visitar \cite{topcoder}.
\end{itemize}

\subsection*{Análisis de complejidad}

Podemos calcular los vectores $enteros$, $rangos$ y $mapIndices$ en $O(n)$, siendo n el tamaño del
vector original del input, si realizamos una sola pasada por ese vector que viene ya ordenado,
identificando las crecientes.

El RMQ tiene un costo de preprocesamiento de orden $O(nlog(n))$ lo cual es simple de verificar, pues
debemos llenar una tabla de dimensiones $N \times log(N)$ de izquierda a derecha y de arriba a abajo.

Por lo tanto la primera parte que incluye la transformación del vector original y el preprocesamiento de RMQ
tiene en su conjunto una complejidad de $O(n + nlog(n))$.

El costo de consulta de RMQ es de orden $O(1)$ según lo explicado en la sección de detalles de implementación.
Para verificar complejidad de RMQ y más información, visitar \cite{topcoder}. Calcular la cantidad de ocurrencias
de los elementos de los bordes del intervalo es una resta de enteros, por lo que tiene orden $O(1)$ también. Por
lo tanto, una vez hecha la primera parte de todo el algoritmo, las consultas del |FV| tienen un orden $O(1)$.

La segunda parte consiste en realizar $q$ consultas, siendo q el del enunciado, por lo que el orden resulta ser
$O(q)$.

Por lo tanto, el costo del algoritmo completo para solucionar el problema dado un input dado es $O(n + nlog(n) + q)$.

\bibliographystyle{acm}
\bibliography{citas}


\end{document}
