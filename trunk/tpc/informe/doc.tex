\documentclass[12pt,spanish]{article}
\usepackage[spanish,activeacute]{babel}
\renewcommand\shorthandsspanish{}
\usepackage[utf8]{inputenc}
\usepackage[T1]{fontenc}
\usepackage{fancyhdr,a4wide,enumerate}
\usepackage{lastpage}
\usepackage{caratula}
\usepackage{amsthm,amsmath,pxfonts,amssymb}
%\usepackage[x11names,rgb]{xcolor}
%\usepackage{tikz}
\usepackage[ruled]{algorithm2e}
\usepackage[protrusion=true,expansion=true]{microtype}

\setlength{\headheight}{16pt}
\pagestyle{fancy}
\lhead{Trabajo Práctico C}
\rhead{Eugenio Costa, Emiliano González, Lisandro Sebrié}
\cfoot{\thepage\ de \pageref{LastPage}}
\renewcommand{\headrulewidth}{0.4pt}
\renewcommand{\footrulewidth}{0.4pt}

\theoremstyle{plain}
\newtheorem*{theorem}{Teorema}
\newtheorem*{lemma}{Lema}
\newtheorem*{proposition}{Proposición}
\newtheorem*{corollary}{Corolario}

\theoremstyle{definition}
\newtheorem*{definition}{Definición}

%\lstset{
%language=C++,
%tabsize=4,
%basicstyle=\footnotesize,
%breaklines=true,
%breakatwhitespace=true,
%}

\dontprintsemicolon
\linesnumberedhidden
\nocaptionofalgo
\SetKwIF{If}{ElseIf}{Else}{si}{luego}{sino si}{sino}{fin si}
\SetKw{Return}{devolver}
\SetKwFor{For}{para}{hacer}{fin para}
\SetKwFor{ForEach}{para cada}{hacer}{fin para}
\SetKwFor{ForAll}{repetir}{veces}{fin repetir}
\SetKwFor{While}{mientras}{hacer}{fin mientras}
\SetKwFor{Rep}{repetir}{veces}{fin repetir}
\SetKwRepeat{Repeat}{repetir}{hasta que}
\SetKw{KwTo}{hasta}
\SetKwInOut{Input}{entrada}
\SetKwInOut{Output}{salida}

\DeclareMathOperator{\set}{set}
\DeclareMathOperator{\sgn}{sgn}
\newcommand{\nat}{\mathbb{N}}
\newcommand{\integ}{\mathbb{Z}}

\begin{document}

\materia{Problemas, Algoritmos y Programación}
\titulo{Trabajo Práctico C}
\grupo{Grupo 14}
\integrante{Eugenio Costa}{839/06}{tutecosta@gmail.com}
\integrante{Emiliano González}{426/06}{xjesse\_jamesx@hotmail.com}
\integrante{Lisandro Sebrié}{358/03}{lsebrie@hotmail.com}

\maketitle

\newpage
\section*{32 -- ``A Needle in the Haystack''}
Un string $s$ ($|s|=n$) ocurre con desplazamiento $p$ en un texto $t$ ($|t|=m$)
si los $n$ caracteres que siguen en $t$ desde la posición $p$ son iguales a los
de $s$. El problema consiste en encontrar todas los desplazamientos en que
ocurre un string $s$ en un texto $t$ (con $m$ en principio de tamaño ilimitado).

Lo resolvemos aplicando el algoritmo de Knuth-Morris-Pratt, que resuelve
precisamente este problema, por lo que su sola implementación redunda en
soluciones válidas, y no se hace necesario preprocesar los datos de entrada o de
salida para adaptarlos a los del problema en cuestión.

\subsection*{Detalles de implementación}

El enunciado sugiere buscar el string a medida que se lee el texto. Para
facilitar la primera implementación leimos tanto $s$ como $t$ completamente
antes de ejecutar KMP (mediante dos {\tt scanf}). Guardando $t$ en un arreglo de
1MB el algoritmo fue aceptado en SPOJ.

Mejoramos luego la implementación leyendo $t$ de a 256 bytes. La solución
también fue aceptada por SPOJ, el uso de memoria fue menor y el tiempo de
cómputo fue casi igual.

\subsection*{Análisis de complejidad}

El algoritmo se compone de dos partes. Primero calcula la tabla de prefijos del
string a buscar, lo que toma tiempo lineal sobre el tamaño de la entrada
($O(m)$). Luego hace la búsqueda propiamente dicha, y toma tiempo lineal sobre
el tamaño del texto ($O(m)$).

La complejidad del algoritmo resulta por tanto de $O(n+m)$.


\end{document}
