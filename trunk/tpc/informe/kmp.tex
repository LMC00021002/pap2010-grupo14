El problema consiste en encontrar todas las ocurrencias de un string $s$
en un texto $t$ ($|s| = n, |t| = m$). Lo resolvemos implementando el algoritmo
de Knuth-Morris-Pratt, que resuelve exactamente este problema, por lo que no
tenemos necesidad de transformar los datos de entrada o de salida.

\subsection*{Detalles de implementación}

Si bien el enunciado sugiere buscar el string a medida que se lee el texto, para
facilitar la implementación leemos tanto $s$ como $t$ mediante dos {\tt scanf}.
Guardamos $t$ en un arreglo de 1MB.

\subsection*{Análisis de complejidad}

La tabla de prefijos se genera en tiempo lineal sobre el tamaño del string a
buscar, $O(n)$; mientras que la búsqueda toma tiempo lineal sobre el tamaño
del texto, $O(m)$. La complejidad del algoritmo resulta entonces de $O(n+m)$.
