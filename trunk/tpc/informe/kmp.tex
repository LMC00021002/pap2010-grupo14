\subsection*{Solución}
El problema nos pide buscar todas las ocurrencias de un patrón en un string dado,
siendo este patrón otro string. Nos piden además devolver la posición en la que
ocurre el patrón dentro del texto.

Un string $s$ ocurre con desplazamiento $p$ en un texto $t$ si los caracteres
de $s$ ocurren en $t$ desde el $p-ésimo$ caracter del texto t. Este desplazamiento
es, por ende, la posición en la que ocurre el patrón dado en el texto. Por lo tanto
debemos encontrar todas los desplazamientos en que ocurre un string $s$ en un texto
$t$ ($|s|=n, |t|=m$, con $m$ y $n$ en principio de valor ilimitado).

Para este problema optamos por utilizar el algoritmo recomendado por el
enunciado: Knuth-Morris-Pratt (KMP), que resuelve precisamente
este problema, por lo que su implementación computa soluciones válidas,
y no se hace necesario preprocesar los datos de entrada o de salida para
adaptar el problema.

\subsection*{Detalles de implementación}

El enunciado sugiere buscar $s$ a medida que se lee $t$. Para facilitar la
primera implementación leimos completamente $s$ y $t$ antes de ejecutar KMP
(con dos {\tt scanf}). Guardando $t$ en un arreglo de 1MB el algoritmo fue
aceptado en SPOJ.

Mejoramos luego la implementación leyendo $t$ de a 256 bytes, de modo que
el tamaño de $t$ no sea una limitante de nuestro algoritmo. La solución
también fue aceptada por SPOJ, el uso de memoria resultó ser menor y el
tiempo de cómputo casi no varió.

\subsection*{Análisis de complejidad}

El algoritmo fue visto en clase. Se compone de dos partes: primero calcula la tabla de prefijos
$s$, lo que toma tiempo lineal sobre el tamaño de la entrada ($O(n)$), luego hace la
búsqueda propiamente dicha, lo que toma tiempo lineal sobre el tamaño del texto ($O(m)$).

Por lo tanto la complejidad del algoritmo resulta ser $O(n + m)$.