El problema consiste en encontrar todas las ocurrencias de un string $s$ (de
largo $n$) en un texto $t$ con largo $m$. Como sugiere el enunciado, lo
resolvemos con el algoritmo de Knuth-Morris-Pratt.


\vspace{.3cm}
The assignment $t = f[t]$ can be done at most $n-1$ times (although the
while loop can iterate more than once on given pass, for example,
if the target is abababb, then f(6)=4, f(f(6))=2, and f(f(f(6)))= 0.
The assignment is done 3 times when s = 6.) It is then easy to see that
the running time of the routine make\_pi is O(m).

Lemma. Let f be a non-negative integer valued function such that
f(1) = 0, and so that f is "continuous upwards", i.e., $f(t+1) \le f(t) + 1$.
Then $A=\sum{(f(s) - f(s+1))+ \le (l-1)}$

Definimos $a+ = max(a,0)$ y $a- = min(a,0)$. (All sums run from 1 to l-1.)
$a = (a+) + (a-)$ para todo número $a$.

\begin{verbatim}
	-f(l) = f(1) - f(l) = \ [ f(s) - f(s+1) ] =
				/__
		__
     A + \ [ f(s) - f(s+1) ]- >= A - (l-1).
		/__
Using f(l) >= 0 and rearranging, we obtain the desired inequality.
QED.

Consider now a given iteration of the 'for' loop with a given value
of s. Let n be the number of times the assignment t = f(t) is done.
Note that f(s+1) is either f^n(s) + 1, or 0 = f^n(s), where f^n denotes the
n-1 fold iterate of f. Since f(t) < t (only proper suffixes are
considered,) we have f^n(s) <= f(s) - n + 1.

Thus, in either case, f(s) - f(s+1) >= n. Now apply the lemma to obtain
the desired bound on the sum of n.
\end{verbatim}

\subsection*{Detalles de implementación}

Si bien el enunciado sugiere buscar el string a medida que se lee el texto, para
facilitar la implementación leemos tanto $s$ como $t$ mediante dos {\tt scanf}.
Guardamos el $t$ en un arreglo de 1MB.

\subsection*{Análisis de complejidad}

La tabla de prefijos se genera en tiempo lineal sobre el tamaño del string a
buscar, $O(n)$; mientras que la búsqueda toma tiempo lineal sobre el tamaño
del texto, $O(n+m)$, lo que redunda en una complejidad total de $O(n+m)$.
