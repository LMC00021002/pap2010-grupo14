\subsection*{Solución}

El problema consiste en buscar todas las ocurrencias de un patrón en un string
dado, siendo este patrón otro string. Nos piden devolver la posición de cada
ocurrencia del patrón en el texto.

Un string $s$ ocurre con desplazamiento $p$ en un texto $t$ si los caracteres
de $s$ ocurren en $t$ desde el $p$-ésimo caracter de $t$. Este desplazamiento
es, por ende, la posición en la que ocurre el patrón dado en el texto.

Optamos por utilizar el algoritmo sugerido por el
enunciado: Knuth-Morris-Pratt (KMP), que resuelve precisamente
este problema, por lo que su implementación computa soluciones válidas,
y no se hace necesario preprocesar los datos de entrada o de salida para
adaptar a este problema.

\subsection*{Detalles de implementación}

El enunciado sugiere buscar $s$ a medida que se lee $t$. Para facilitar la
primera implementación leimos completamente $s$ y $t$ antes de ejecutar KMP
(con dos {\tt scanf}). Guardando $t$ en un arreglo de 1MB el algoritmo fue
aceptado en SPOJ.

Mejoramos luego la implementación leyendo $t$ de a 256 bytes, de modo que
el tamaño de $t$ no sea una limitante de nuestro algoritmo. La solución
también fue aceptada por SPOJ, el uso de memoria resultó ser menor y el
tiempo de cómputo casi no varió.

\subsection*{Análisis de complejidad}

Este algoritmo fue visto en clase. Se compone de dos partes: primero calcula
la tabla de prefijos de $s$, lo que toma tiempo lineal sobre el tamaño de la
entrada: $O(|s|)$. Luego hace la búsqueda propiamente dicha, lo que toma
tiempo lineal sobre el tamaño del texto $t$: $O(|t|)$. Por lo tanto la
complejidad del algoritmo resulta de $O(|s| + |t|)$.
