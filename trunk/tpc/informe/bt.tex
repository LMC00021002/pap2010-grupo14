El problema se define como el procesamiento de una tabla de planes para
prefijos de números telefónicos. Para modelarlo utilizamos un trie que
toma como clave el número de prefijo y como significado el nombre del
plan. Cada nodo tiene punteros a diez hijos, una clave, y un puntero a string
(significado).

Por otro lado tenemos un conjunto que contiene todos los {\sl bill plans}
que van apareciendo, lo que simplifica la comparación entre {\sl billing
plans} debido a que nos permite realizar comparaciones a nivel de punteros.

\subsection*{Detalles de implementación}

El algoritmo consta de 3 fases: en la primera se van agregando los prefijos
al diccionario, en la segunda se comprimen aquellos nodos que tienen a todos sus
hijos con igual {\sl billing plan}, y en la tercer etapa se imprimen los prefijos
resultantes del proceso.\\

Las inserciones se pueden dividir en tres casos:

\begin{enumerate}
  \item Al descender a través de la rama del árbol correspondiente se
  encuentra con una hoja, lo que significa que ya existe un plan que tiene
  un código que es prefijo del que se intenta insertar. Los códigos que se
  necesitan ya están ocupados, y no se agrega el plan.

  \item Descendiendo se intenta definir un significado en un nodo que
  tiene hijos. En este caso el algoritmo debe completar todos los nodos que
  corresponda, tratando de expandirse a través de todos los hijos que no
  estén completos. Puede llegar a tener que completar la rama del hijo que
  contenía un plan definido con anterioridad.

  \item Al insertar se llega a una primer hoja, sin que tenga hijos el nodo
  donde se define el significado. En este único caso se realiza la inserción
  normal en el diccionario.
\end{enumerate}

Para determinar cual es el próximo código a insertar dado un {\sl billing
plan} se realiza una iteración dentro del intervalo de los códigos definidos
para el plan. Pero para que no atraviese por n valores, todos sufijos de un
mismo prefijo, se chequea si el siguiente código termina con '0', y tratamos
de reducir cortando el último caracter, siempre que el código resultante
no contenga ya algún otro código que tenga a este como prefijo. Repetimos
el procedimiento mientras se pueda, y la inserción se realiza al nivel más
alto posible del árbol.

En general durante el algoritmo tratamos los códigos como números
enteros. Para avanzar de un código a otro se computa una suma. La ventaja es
que, dados $(a,b)$, definir el tamaño original del intervalo de un plan se
computa como una suma entre los dígitos de $b$ y $a$ con los últimos $k$
dígitos en 0, donde $k$ es la cantidad de cifras de $b$. Surgió con esto
el problema de trabajar con códigos que comiencen con ceros, debemos llevar
la cuenta de cuántos ceros tiene al inicio cada código, para ubicarlos
como prefijos al imprimir el árbol.\\

En la segunda fase se comprime el árbol. Para saber que un nodo se puede
comprimir, todos sus hijos deben pertenecer al mismo {\sl billing plan}. Este
proceso se puede realizar de manera recursiva: en cada paso primero se
procesan todos los hijos de un nodo, y luego él mismo. La recorrida a
través del árbol se realiza mediante un DFS, que retornará a un nodo el
nombre del plan que comparten sus hijos si es que existe alguno. No ofrece
ventaja realizar la compresión a medida que se realizan las inserciones,
porque si el nodo era candidato a comprimirse ya estaba completo, y no se
iban a realizar allí nuevas inserciones.\\

En la tercera y última fase se realiza la impresión de los códigos nuevos
obtenidos, lo que se reduce a recorrer todas las claves definidas en el
diccionario e imprimirlas. Para lograr imprimir el número de líneas de
la nueva tabla mantenemos una variable que cuenta las claves definidas en
cada momento que no sean inválidas. Cada vez que se agrega una clave se
la incrementa, mientras que cuando se comprime se la decrementa en nueve
(pues se borran diez hojas y se inserta una).

Por último, necesitamos utilizar {\tt long long int} para almacenar los
prefijos, pues pueden tener una longitud máxima de 11 caracteres, números
que no entran en 32 bits. Debimos también definir el pasaje de {\tt long
long int} a {\tt char} y viceversa, dado que {\tt atoi} convierte de {\tt
char} a {\tt int}. Lo mismo ocurrió con {\tt sprintf} (de arreglo a número)
y {\tt pow10} (por alguna razón {\tt pow} no funcionaba correctamente).

\begin{algorithm}[H]
\linesnumbered
\caption{artic\_point\_DFS($v,G$)}
\Input{vértice $v$, grafo $G(V,E)$}

\vspace{0.4cm}
\For{$x$ tal que $\{v,x\} \in E$}{
        \If{$x.dfs\_num = -1$}{
                $v.num\_children \gets v.num\_children+1$\;
                $stack.push(\{v,x\})$\;
                $artic\_point\_DFS(x,G)$
        } \lElse{\If{$x.dfs\_num < v.dfs\_num - 1$}{
                $stack.push(\{v,x\})$\;
        }}
}
\end{algorithm}

\subsection*{Análisis de complejidad}

Sea k el largo de las ramas del árbol y t el número de caracteres del alfabeto del árbol.

El orden de complejidad del algoritmo se puede determinar por cada una de las operaciones que realizamos
en cada fase. 

En la primera fase el algoritmo determinante es el de agregar un código. 
En la fase de inserción el debemos calcular la cantidad de ceros que tiene el código. 
El orden de éste es O(k). Luego este algoritmo va a ser utilizado durante toda la primera fase.

El caso en el que la inserción deviene en ingresar un único código el
orden del algoritmo es $O(k^2)$, que es la profundidad del árbol, que es la
máxima cantidad de nodos que va a atravesar por el orden de cantidad de ceros.

En cambio cuando agregar un código se convierte en completar los niveles
subsiguientes porque el nodo que se intentaba definir es prefijo de uno ya
existente, entonces se debe definir cada una de las ramas de los hijos y
explorarlas hacia abajo cuanto haga falta. El orden para el peor caso que se
obtiene del algoritmo es $O(k*t^k)$, que está dado por la profundidad de cada
rama y la cantidad de hijos que puede tener un nodo y la cantidad de ceros.

Por último queda el caso donde la inserción se ve cortada porque no
se llega a consumir toda la cadena a insertar y se encuentra un código
definido antes. El orden para este caso se puede acotar por $O(k^2)$, al igual
que el primero.

En la segunda fase se realiza la compresión de las claves. El peor caso es
donde el diccionario está definido en todas las hojas de máxima profundidad
con un mismo plan. En ese caso tiene que realizar un chequeo sobre cada uno
de los nodos del árbol. El orden al igual que el de completar es de $O(t^k)$.

Por último queda la tercera fase que es la de recorrer todas las claves para
poder imprimirlas. Este tiene que pasar por todos los nodos del árbol por lo que
el orden del mismo es $O(t^k)$.
