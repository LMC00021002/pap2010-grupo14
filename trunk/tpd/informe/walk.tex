\subsection*{Problema y modelo}

El problema trata sobre dos personas Alice y Bob. Viven en una zona montañosa, y alice quiere ir a visitar
a Bob escalando y descendiendo lo mínimo posible. Nos proveen de una descripcion del terreno con un mapa
de alturas. Este mapa está compuesto de varios polígonos. Cada polígono nos informa cuál es la altura del
terreno sobre su perímetro. Los polígonos no se intersecan entre sí ni consigo mismos.

La descripción que nos dan del terreno es parcial, pues no sabemos qué alturas hay en el terreno en medio
de cualquier par de curvas de altura de polígonos distintos. Por esto, no podemos saber exactamente cuánto
escala o desciende Alice en la práctica, pero debemos calcular el mínimo suponiendo un escenario óptimo.

Alice se encuentra en las coordenadas $(0, 0)$ del mapa, y Bob se encuentra en las coordenadas
$(100000, 0)$.

Presentamos un modelo simple en el que representamos los polígonos como un conjunto de segmentos en los
ejes XY. Cada segmento tiene un valor de altura dado por el enunciado. Alice y Bob se encuentran en las
coordenadas dadas por el enunciado.

\subsection*{Solución}

La idea de la solución que presentamos se basa en los siguientes principios:
\begin{itemize}
\item como dice el enunciado, no existe intersección entre polígonos, de modo que dados dos polígonos
hay solamente dos casos: uno contiene al otro o son totalmente disjuntos.
\item discriminar los polígonos según si Alice tiene que cruzarlos obligatoriamente o si puede
evitarlos, de modo que Alice solamente cruza aquellos que necesariamente debe cruzar para llegar a Bob.
Un polígono es evitable si existe algún camino entre alice y bob que no lo cruce.
\item dado que no se puede saber cuáles son las alturas del terreno que se ubica entre dos curvas de
altura de distintos polígonos, la altura del perímetro de cada polígono es independiente de la de los
demás.
\end{itemize}

Cada vez que Alice cruza un polígono la cantidad de altura escalada y bajada aumenta, lo que significa que
estos valores dependen de la cantidad y el orden de los polígonos que cruza Alice. Logramos una cantidad
mínima de polígonos cruzados si respetamos el segundo principio.

Podemos clasificar todos los polígonos como:
\begin{itemize}
\item contiene solamente a Alice: para llegar a Bob, Alice va a tener que cruzarlo, por lo que estos
polígonos no son evitables.
\item contiene solamente a Bob: Alice esta fuera de este polígono, y necesita cruzarlo, pues sino no podrá
llegar a Bob. Estos polígonos no son evitables.
\item no contiene a Alice ni a Bob: este polígono es evitable, es decir, Alice no necesita cruzarlo para
llegar a Bob, pues suponiendo que Alice cruza completamente el polígono (entra y luego sale) y llega a
algún lugar, siempre existe un camino que lo rodea y llega al mismo lugar sin cruzar otro polígono.
\item contiene a ambos: Alice se puede mantener dentro de este polígono y no necesita cruzarlo, de modo
que también es evitable. En caso de que Alice salga de este polígono, luego para llegar a Bob deberá
cruzarlo. Si consideramos el estado incial de Alice (justo antes de cruzar este polígono) y su estado
final (justo después de cruzarlo nuevamente) podemos decir que siempre existe un camino entre estos estados
inicial y final que no lo cruza nunca ni atravieza otros polígonos.
\end{itemize}

La solución al problema presentada ignora completamente los polígonos evitables, dado que la altura de un
polígono es independiente de la de los demás. Consideremos entonces los polígonos no evitables.

Notemos que se puede definir un orden total entre los polígonos que contienen solamente a Alice, y otro
para los que contienen solamente a Bob. Consideramos ordenado a una secuencia de polígonos tal que el
primero es contenido por todos los siguientes, y el último contiene a todos los anteriores. Esto nos
permite afirmar que existe una única solución para cada escenario del problema, pues hay una sola forma de
cruzar los polígonos que contienen a Alice (de ``adentro'' hacia ``afuera'') y una sola forma de cruzar los
que contienen a Bob(de ``afuera'' hacia ``adentro''). Esta idea justifica la solución presentada.

Dado un escenario, para cada polígono verificamos si contiene a Alice y no a Bob o si contiene a Bob y no
a Alice (ver sección ``Determinación de pertenencia de un punto a un polígono''). En estos casos traducimos
el polígono a un par $(x_1, x_2, h)$ tal que $x_1$ y $x_2$ son dos puntos distintos de intersección con el
eje X, $x_1$ está a la izquierda del punto que verificamos, $x_2$ a la derecha y $h$ es la altura de las
curvas de altura del polígono. Podemos abstraernos de esta forma gracias a que los polígonos están
contenidos o totalmente disjuntos, de modo que para definir un orden total entre los que están contenidos
es suficiente con saber cualquier par de puntos de sus perímetros.

\subsection*{Detalles de Implementación}

\subsection*{Análisis de complejidad}