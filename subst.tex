El problema consiste en determinar la cantidad de substrings distintos que
pueden formarse con una cadena de caracteres. El input contiene un conjunto de
cadenas, y se debe calcular la cantidad de substrings distintos para cada una.

Si tenemos una cadena $s$, podemos obtener un substring $s_{ij}$ (con $1
\le i \le j \le |s|$) conociendo el sufijo $s_i$, y tomando luego su prefijo
$s_{i_j}$. Además, resulta fácil ver que si tenemos dos sufijos $s_{k1}$ y
$s_{k2}$ ($1 \le k1 \le k2 \le |s|$), si la primer letra del primer sufijo es
distinta a la primer letra del segundo sufijo, entonces los prefijos de esos
sufijos serán todos diferentes entre sí. En cambio si existe un prefijo $p$
en común, los sufijos tendrán $|p|$ prefijos en común, lo que significa
que ambos comparten $|p|$ substrings. Por tanto, la cantidad de subcadenas
distintas que ambos sufijos generan será igual a $|s_{k1}| + |s_{k1}| -
|p|$. Esta es la idea del algoritmo que usamos para resolver este problema.

% FIXME: Explicaría más esto que sigue.
La estructura de datos que facilita la implementación de la idea es {\sl suffix
array}. Calculamos el {\sl suffix array} de la cadena, y luego el {\sl Longest
Common Prefix} (LCP). El objetivo es utilizar el LCP array para calcular la
cantidad de substrings que comparten dos sufijos.

Utilizamos una implementación de {\sl suffix array} basada en la presentada
en \cite{kasai}. La idea del algoritmo es que puede ordenarse todos los
sufijos comparando sólo 2 caracteres por vez. En cada iteración aprovecha
la información obtenida en los iteraciones anteriores, de forma que en
cada paso conoce el doble de la información que en el anterior, logrando
realizar el ordenamiento en $\log(n)$ iteraciones.

% FIXME: Sadakane es el paper de la ACM?
% FIXME: Explicar mejor?
Para calcular el LCP utilizamos el algoritmo de Sadakane que calcula el LCP
correspondiente a un suffix array dado en tiempo lineal sobre la cantidad de
sufijos. Su concepto principal es aprovechar que si el sufijo $s_{k1}$ y el
$s_{k2}$ tienen en común un prefijo $p$, los sufijos $s_{k1+1}$ y $s_{k2+1}$
tienen en común un prefijo $p'$ de longitud $|p'| \ge |p|-1$.

Finalmente, para devolver la cantidad de substrings distintos,
calculamos la cantidad total de substrings de la cadena, que es
$\displaystyle\frac{|s|(|s|+1)}{2}$, y le restamos la sumatoria del arreglo
de LCP, que se corresponde con la cantidad total de substrings repetidos.

\subsection*{Detalles de implementación}
% FIXME: COMPLETAR CON LAS ESTRUCTURAS USADAS EN LA IMPLEMENTACION

\subsection*{Análisis de complejidad}

Ambos algoritmos fueron vistos en clase. El orden para calcular el {\sl suffix
array} es $O(n\log^2(n))$, y calcular su LCP toma tiempo $O(n)$. Recorrer el
LCP array tiene un orden de $O(n)$, por lo que el orden total del algoritmo
resulta $O(n\log^2(n))$.
